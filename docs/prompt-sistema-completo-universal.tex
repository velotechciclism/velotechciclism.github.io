\documentclass[12pt,a4paper]{article}
\usepackage[utf8]{inputenc}
\usepackage[T1]{fontenc}
\usepackage[portuguese]{babel}
\usepackage{geometry}
\usepackage{listings}
\usepackage{xcolor}
\usepackage{hyperref}
\usepackage{tcolorbox}
\usepackage{enumitem}
\usepackage{fancyhdr}
\usepackage{titlesec}

\geometry{margin=2.5cm}

\definecolor{codegreen}{rgb}{0,0.6,0}
\definecolor{codegray}{rgb}{0.5,0.5,0.5}
\definecolor{codepurple}{rgb}{0.58,0,0.82}
\definecolor{backcolour}{rgb}{0.95,0.95,0.92}
\definecolor{primary}{RGB}{34,139,34}

\lstdefinestyle{mystyle}{
    backgroundcolor=\color{backcolour},
    commentstyle=\color{codegreen},
    keywordstyle=\color{magenta},
    numberstyle=\tiny\color{codegray},
    stringstyle=\color{codepurple},
    basicstyle=\ttfamily\footnotesize,
    breakatwhitespace=false,
    breaklines=true,
    captionpos=b,
    keepspaces=true,
    numbers=left,
    numbersep=5pt,
    showspaces=false,
    showstringspaces=false,
    showtabs=false,
    tabsize=2,
    frame=single
}
\lstset{style=mystyle}

\pagestyle{fancy}
\fancyhf{}
\rhead{Sistema Web Completo Universal}
\lhead{Documentação Técnica}
\rfoot{Página \thepage}

\title{\textbf{PROMPT UNIVERSAL: Sistema Web Completo}\\
\large Plataforma E-commerce com Autenticação, Carrinho Persistente,\\
Histórico de Pedidos, Chatbot Inteligente e Suporte Multilíngue\\[1cm]
\normalsize Aplicável a Qualquer Segmento de Negócio}
\author{Documentação Técnica para Replicação}
\date{\today}

\begin{document}

\maketitle
\tableofcontents
\newpage

%==============================================================================
\section{Introdução e Visão Geral}
%==============================================================================

Este documento apresenta um guia completo para construção de uma plataforma web moderna, escalável e segura. O sistema descrito pode ser adaptado para qualquer tipo de negócio: e-commerce, serviços, educação, saúde, restaurantes, imobiliárias, entre outros.

\subsection{Capacidades do Sistema}

\begin{tcolorbox}[colback=green!5!white,colframe=primary,title=Funcionalidades Implementadas]
\begin{enumerate}[leftmargin=*]
    \item \textbf{Autenticação Robusta}: Registro, login e gestão de perfis de usuários
    \item \textbf{Carrinho Persistente}: Sincronizado com banco de dados por usuário
    \item \textbf{Sistema de Pedidos}: Histórico completo com itens e status
    \item \textbf{Chatbot Inteligente (RAG)}: Assistente com busca vetorial semântica
    \item \textbf{Internacionalização}: Suporte a múltiplos idiomas
    \item \textbf{Temas}: Modo claro e escuro
    \item \textbf{Design Responsivo}: Adaptável a todos os dispositivos
    \item \textbf{Segurança}: Row Level Security (RLS) em todas as tabelas
\end{enumerate}
\end{tcolorbox}

\subsection{Stack Tecnológico}

\begin{itemize}
    \item \textbf{Frontend}: React 18 + TypeScript + Vite
    \item \textbf{Estilização}: Tailwind CSS + shadcn/ui
    \item \textbf{Backend}: Supabase (PostgreSQL + Edge Functions)
    \item \textbf{IA}: Lovable AI Gateway (Gemini/GPT)
    \item \textbf{Busca Vetorial}: pgvector para RAG
    \item \textbf{Roteamento}: React Router DOM
    \item \textbf{Estado}: React Context API + React Query
\end{itemize}

%==============================================================================
\section{Prompt Universal para Criação do Sistema}
%==============================================================================

\begin{tcolorbox}[colback=blue!5!white,colframe=blue!75!black,title=PROMPT COMPLETO PARA IA]
\small
\textbf{Crie uma plataforma web completa para [NOME DO NEGÓCIO], um(a) [TIPO DE NEGÓCIO] especializado(a) em [SEGMENTO].}

\textbf{REQUISITOS OBRIGATÓRIOS:}

\textbf{1. ARQUITETURA E TECNOLOGIAS}
\begin{itemize}[leftmargin=*,nosep]
    \item React 18 com TypeScript e Vite
    \item Tailwind CSS com design system customizado
    \item Componentes shadcn/ui
    \item Supabase para backend (PostgreSQL)
    \item React Router para navegação SPA
\end{itemize}

\textbf{2. SISTEMA DE AUTENTICAÇÃO}
\begin{itemize}[leftmargin=*,nosep]
    \item Página /auth com formulários de login e registro
    \item Campos: email, senha, nome, telefone (opcional), endereço (opcional)
    \item Tabela profiles vinculada ao auth.users via trigger
    \item Context AuthProvider global com useAuth hook
    \item Proteção de rotas para usuários autenticados
    \item Mensagens de erro amigáveis (email já cadastrado, senha inválida)
\end{itemize}

\textbf{3. CARRINHO DE COMPRAS PERSISTENTE}
\begin{itemize}[leftmargin=*,nosep]
    \item Tabela cart\_items com: user\_id, product\_id, product\_name, product\_image, product\_price, quantity
    \item Sincronização automática com banco quando usuário logado
    \item Funcionamento local quando não logado
    \item Funções: addItem, removeItem, updateQuantity, clearCart
    \item Context CartProvider com useCart hook
\end{itemize}

\textbf{4. SISTEMA DE PEDIDOS}
\begin{itemize}[leftmargin=*,nosep]
    \item Tabela orders: user\_id, total, status, payment\_method, shipping\_address
    \item Tabela order\_items: order\_id, product\_id, product\_name, product\_price, quantity
    \item Página /orders com histórico do usuário
    \item Função checkout que cria pedido e limpa carrinho
    \item Status: pending, processing, shipped, delivered, cancelled
\end{itemize}

\textbf{5. CHATBOT INTELIGENTE COM RAG}
\begin{itemize}[leftmargin=*,nosep]
    \item Extensão pgvector habilitada
    \item Tabela entity\_embeddings com campo embedding vector(1536)
    \item Função search\_similar\_entities para busca semântica
    \item Edge Function /chatbot usando Lovable AI Gateway
    \item Widget flutuante em todas as páginas
    \item Histórico de conversas persistido
\end{itemize}

\textbf{6. INTERNACIONALIZAÇÃO (i18n)}
\begin{itemize}[leftmargin=*,nosep]
    \item Arquivos JSON: locales/pt-br.json, locales/en.json
    \item Context LanguageProvider com useLanguage hook
    \item Função t() para tradução de chaves
    \item Seletor de idioma no header
    \item Persistência da preferência em localStorage
\end{itemize}

\textbf{7. SISTEMA DE TEMAS}
\begin{itemize}[leftmargin=*,nosep]
    \item Context ThemeProvider com useTheme hook
    \item Tokens CSS em index.css para :root e .dark
    \item Toggle de tema no header
    \item Persistência em localStorage
    \item Respeitar preferência do sistema (prefers-color-scheme)
\end{itemize}

\textbf{8. PÁGINAS NECESSÁRIAS}
\begin{itemize}[leftmargin=*,nosep]
    \item / - Home com hero, categorias, produtos destaque, depoimentos
    \item /products - Catálogo com filtros (categoria, preço, marca)
    \item /products/:id - Detalhe do produto com especificações
    \item /cart - Carrinho com resumo e checkout
    \item /auth - Login e registro
    \item /orders - Histórico de pedidos (protegida)
    \item /contact - Formulário de contato
    \item /help - FAQ e suporte
    \item /* - Página 404 NotFound
\end{itemize}

\textbf{9. SEGURANÇA (RLS)}
\begin{itemize}[leftmargin=*,nosep]
    \item Todas as tabelas com Row Level Security habilitado
    \item Políticas: usuários só acessam próprios dados (auth.uid() = user\_id)
    \item Trigger para criar profile automaticamente no registro
    \item Validação de inputs com Zod
\end{itemize}

\textbf{10. DESIGN E UX}
\begin{itemize}[leftmargin=*,nosep]
    \item Layout responsivo (mobile-first)
    \item Header com navegação, busca, carrinho, perfil
    \item Footer com links, contato, redes sociais
    \item Loading states e skeleton loaders
    \item Toast notifications para feedback
    \item Animações suaves com Framer Motion (opcional)
\end{itemize}
\end{tcolorbox}

%==============================================================================
\section{Estrutura do Banco de Dados}
%==============================================================================

\subsection{Schema Completo SQL}

\begin{lstlisting}[language=SQL,caption=Migração Completa do Banco de Dados]
-- ============================================
-- 1. EXTENSOES NECESSARIAS
-- ============================================
CREATE EXTENSION IF NOT EXISTS vector;

-- ============================================
-- 2. TABELA DE PERFIS DE USUARIOS
-- ============================================
CREATE TABLE public.profiles (
  id UUID NOT NULL DEFAULT gen_random_uuid() PRIMARY KEY,
  user_id UUID NOT NULL UNIQUE,
  name TEXT NOT NULL,
  phone TEXT,
  address TEXT,
  created_at TIMESTAMP WITH TIME ZONE NOT NULL DEFAULT now(),
  updated_at TIMESTAMP WITH TIME ZONE NOT NULL DEFAULT now()
);

ALTER TABLE public.profiles ENABLE ROW LEVEL SECURITY;

CREATE POLICY "Users can view own profile" 
  ON public.profiles FOR SELECT 
  USING (auth.uid() = user_id);

CREATE POLICY "Users can update own profile" 
  ON public.profiles FOR UPDATE 
  USING (auth.uid() = user_id);

CREATE POLICY "Users can insert own profile" 
  ON public.profiles FOR INSERT 
  WITH CHECK (auth.uid() = user_id);

-- Trigger para criar perfil automaticamente
CREATE OR REPLACE FUNCTION public.handle_new_user()
RETURNS TRIGGER AS $$
BEGIN
  INSERT INTO public.profiles (user_id, name)
  VALUES (NEW.id, COALESCE(NEW.raw_user_meta_data->>'name', 'Usuario'));
  RETURN NEW;
END;
$$ LANGUAGE plpgsql SECURITY DEFINER;

CREATE TRIGGER on_auth_user_created
  AFTER INSERT ON auth.users
  FOR EACH ROW EXECUTE FUNCTION public.handle_new_user();

-- ============================================
-- 3. TABELA DE ITENS DO CARRINHO
-- ============================================
CREATE TABLE public.cart_items (
  id UUID NOT NULL DEFAULT gen_random_uuid() PRIMARY KEY,
  user_id UUID NOT NULL,
  product_id TEXT NOT NULL,
  product_name TEXT NOT NULL,
  product_image TEXT,
  product_price NUMERIC NOT NULL,
  quantity INTEGER NOT NULL DEFAULT 1,
  created_at TIMESTAMP WITH TIME ZONE NOT NULL DEFAULT now(),
  updated_at TIMESTAMP WITH TIME ZONE NOT NULL DEFAULT now()
);

ALTER TABLE public.cart_items ENABLE ROW LEVEL SECURITY;

CREATE POLICY "Users can view own cart" 
  ON public.cart_items FOR SELECT 
  USING (auth.uid() = user_id);

CREATE POLICY "Users can insert own cart" 
  ON public.cart_items FOR INSERT 
  WITH CHECK (auth.uid() = user_id);

CREATE POLICY "Users can update own cart" 
  ON public.cart_items FOR UPDATE 
  USING (auth.uid() = user_id);

CREATE POLICY "Users can delete own cart" 
  ON public.cart_items FOR DELETE 
  USING (auth.uid() = user_id);

-- ============================================
-- 4. TABELA DE PEDIDOS
-- ============================================
CREATE TABLE public.orders (
  id UUID NOT NULL DEFAULT gen_random_uuid() PRIMARY KEY,
  user_id UUID NOT NULL,
  total NUMERIC NOT NULL,
  status TEXT NOT NULL DEFAULT 'pending',
  payment_method TEXT,
  shipping_address TEXT,
  created_at TIMESTAMP WITH TIME ZONE NOT NULL DEFAULT now(),
  updated_at TIMESTAMP WITH TIME ZONE NOT NULL DEFAULT now()
);

ALTER TABLE public.orders ENABLE ROW LEVEL SECURITY;

CREATE POLICY "Users can view own orders" 
  ON public.orders FOR SELECT 
  USING (auth.uid() = user_id);

CREATE POLICY "Users can create own orders" 
  ON public.orders FOR INSERT 
  WITH CHECK (auth.uid() = user_id);

CREATE POLICY "Users can update own orders" 
  ON public.orders FOR UPDATE 
  USING (auth.uid() = user_id);

-- ============================================
-- 5. TABELA DE ITENS DO PEDIDO
-- ============================================
CREATE TABLE public.order_items (
  id UUID NOT NULL DEFAULT gen_random_uuid() PRIMARY KEY,
  order_id UUID NOT NULL REFERENCES public.orders(id),
  product_id TEXT NOT NULL,
  product_name TEXT NOT NULL,
  product_image TEXT,
  product_price NUMERIC NOT NULL,
  quantity INTEGER NOT NULL,
  created_at TIMESTAMP WITH TIME ZONE NOT NULL DEFAULT now()
);

ALTER TABLE public.order_items ENABLE ROW LEVEL SECURITY;

CREATE POLICY "Users can view own order items" 
  ON public.order_items FOR SELECT 
  USING (EXISTS (
    SELECT 1 FROM orders 
    WHERE orders.id = order_items.order_id 
    AND orders.user_id = auth.uid()
  ));

CREATE POLICY "Users can insert own order items" 
  ON public.order_items FOR INSERT 
  WITH CHECK (EXISTS (
    SELECT 1 FROM orders 
    WHERE orders.id = order_items.order_id 
    AND orders.user_id = auth.uid()
  ));

-- ============================================
-- 6. TABELA DE EMBEDDINGS PARA RAG
-- ============================================
CREATE TABLE public.entity_embeddings (
  id UUID NOT NULL DEFAULT gen_random_uuid() PRIMARY KEY,
  entity_id TEXT NOT NULL,
  entity_name TEXT NOT NULL,
  entity_description TEXT,
  entity_category TEXT,
  entity_price NUMERIC,
  embedding vector(1536),
  created_at TIMESTAMP WITH TIME ZONE NOT NULL DEFAULT now(),
  updated_at TIMESTAMP WITH TIME ZONE NOT NULL DEFAULT now()
);

ALTER TABLE public.entity_embeddings ENABLE ROW LEVEL SECURITY;

CREATE POLICY "Embeddings are public" 
  ON public.entity_embeddings FOR SELECT 
  USING (true);

-- Indice para busca vetorial
CREATE INDEX ON public.entity_embeddings 
  USING ivfflat (embedding vector_cosine_ops) 
  WITH (lists = 100);

-- Funcao de busca semantica
CREATE OR REPLACE FUNCTION public.search_similar_entities(
  query_embedding vector,
  match_threshold FLOAT DEFAULT 0.5,
  match_count INT DEFAULT 5
)
RETURNS TABLE (
  entity_id TEXT,
  entity_name TEXT,
  entity_description TEXT,
  entity_category TEXT,
  entity_price NUMERIC,
  similarity FLOAT
)
LANGUAGE plpgsql
AS $$
BEGIN
  RETURN QUERY
  SELECT
    e.entity_id,
    e.entity_name,
    e.entity_description,
    e.entity_category,
    e.entity_price,
    1 - (e.embedding <=> query_embedding) AS similarity
  FROM entity_embeddings e
  WHERE 1 - (e.embedding <=> query_embedding) > match_threshold
  ORDER BY e.embedding <=> query_embedding
  LIMIT match_count;
END;
$$;

-- ============================================
-- 7. TABELAS DE CHAT
-- ============================================
CREATE TABLE public.chat_conversations (
  id UUID NOT NULL DEFAULT gen_random_uuid() PRIMARY KEY,
  user_id UUID,
  session_id TEXT NOT NULL,
  created_at TIMESTAMP WITH TIME ZONE NOT NULL DEFAULT now()
);

ALTER TABLE public.chat_conversations ENABLE ROW LEVEL SECURITY;

CREATE POLICY "Anyone can create conversations" 
  ON public.chat_conversations FOR INSERT 
  WITH CHECK (true);

CREATE POLICY "Users can view own conversations" 
  ON public.chat_conversations FOR SELECT 
  USING (auth.uid() = user_id OR user_id IS NULL);

CREATE TABLE public.chat_messages (
  id UUID NOT NULL DEFAULT gen_random_uuid() PRIMARY KEY,
  conversation_id UUID NOT NULL REFERENCES public.chat_conversations(id),
  role TEXT NOT NULL,
  content TEXT NOT NULL,
  created_at TIMESTAMP WITH TIME ZONE NOT NULL DEFAULT now()
);

ALTER TABLE public.chat_messages ENABLE ROW LEVEL SECURITY;

CREATE POLICY "Anyone can insert messages" 
  ON public.chat_messages FOR INSERT 
  WITH CHECK (true);

CREATE POLICY "Users can view conversation messages" 
  ON public.chat_messages FOR SELECT 
  USING (EXISTS (
    SELECT 1 FROM chat_conversations 
    WHERE chat_conversations.id = chat_messages.conversation_id 
    AND (chat_conversations.user_id = auth.uid() OR chat_conversations.user_id IS NULL)
  ));

-- ============================================
-- 8. FUNCAO DE ATUALIZACAO DE TIMESTAMP
-- ============================================
CREATE OR REPLACE FUNCTION public.update_updated_at_column()
RETURNS TRIGGER AS $$
BEGIN
  NEW.updated_at = now();
  RETURN NEW;
END;
$$ LANGUAGE plpgsql;

-- Aplicar triggers de updated_at
CREATE TRIGGER update_profiles_updated_at
  BEFORE UPDATE ON public.profiles
  FOR EACH ROW EXECUTE FUNCTION public.update_updated_at_column();

CREATE TRIGGER update_cart_items_updated_at
  BEFORE UPDATE ON public.cart_items
  FOR EACH ROW EXECUTE FUNCTION public.update_updated_at_column();

CREATE TRIGGER update_orders_updated_at
  BEFORE UPDATE ON public.orders
  FOR EACH ROW EXECUTE FUNCTION public.update_updated_at_column();
\end{lstlisting}

%==============================================================================
\section{Arquitetura de Contextos React}
%==============================================================================

\subsection{AuthContext - Autenticação}

\begin{lstlisting}[language=JavaScript,caption=Hook useAuth Completo]
// src/hooks/useAuth.ts
import { useState, useEffect } from 'react';
import { User, Session } from '@supabase/supabase-js';
import { supabase } from '@/integrations/supabase/client';

interface Profile {
  id: string;
  user_id: string;
  name: string;
  phone: string | null;
  address: string | null;
}

export function useAuth() {
  const [user, setUser] = useState<User | null>(null);
  const [session, setSession] = useState<Session | null>(null);
  const [profile, setProfile] = useState<Profile | null>(null);
  const [isLoading, setIsLoading] = useState(true);

  useEffect(() => {
    // Listener PRIMEIRO
    const { data: { subscription } } = supabase.auth.onAuthStateChange(
      (event, session) => {
        setSession(session);
        setUser(session?.user ?? null);
        
        if (session?.user) {
          setTimeout(() => fetchProfile(session.user.id), 0);
        } else {
          setProfile(null);
        }
      }
    );

    // DEPOIS verificar sessao existente
    supabase.auth.getSession().then(({ data: { session } }) => {
      setSession(session);
      setUser(session?.user ?? null);
      if (session?.user) fetchProfile(session.user.id);
      setIsLoading(false);
    });

    return () => subscription.unsubscribe();
  }, []);

  const fetchProfile = async (userId: string) => {
    const { data } = await supabase
      .from('profiles')
      .select('*')
      .eq('user_id', userId)
      .maybeSingle();
    setProfile(data);
  };

  const register = async (email: string, name: string, password: string) => {
    const { data, error } = await supabase.auth.signUp({
      email,
      password,
      options: {
        emailRedirectTo: `${window.location.origin}/`,
        data: { name }
      }
    });
    if (error) throw error;
    return data;
  };

  const login = async (email: string, password: string) => {
    const { data, error } = await supabase.auth.signInWithPassword({
      email,
      password,
    });
    if (error) throw error;
    return data;
  };

  const logout = async () => {
    await supabase.auth.signOut();
    setUser(null);
    setSession(null);
    setProfile(null);
  };

  return {
    user,
    profile,
    session,
    isLoading,
    isAuthenticated: !!session,
    register,
    login,
    logout,
  };
}
\end{lstlisting}

\subsection{CartContext - Carrinho Persistente}

\begin{lstlisting}[language=JavaScript,caption=Hook useCartPersistence]
// src/hooks/useCartPersistence.ts
import { useState, useEffect, useCallback } from 'react';
import { supabase } from '@/integrations/supabase/client';
import { useAuthContext } from '@/context/AuthContext';

export function useCartPersistence() {
  const { user, isAuthenticated } = useAuthContext();
  const [items, setItems] = useState([]);
  const [isLoading, setIsLoading] = useState(true);

  const loadCartItems = useCallback(async () => {
    if (!isAuthenticated || !user) {
      setItems([]);
      setIsLoading(false);
      return;
    }

    const { data } = await supabase
      .from('cart_items')
      .select('*')
      .eq('user_id', user.id)
      .order('created_at');

    setItems(data?.map(item => ({
      id: item.product_id,
      name: item.product_name,
      image: item.product_image,
      price: Number(item.product_price),
      quantity: item.quantity,
    })) || []);
    setIsLoading(false);
  }, [isAuthenticated, user]);

  useEffect(() => {
    loadCartItems();
  }, [loadCartItems]);

  const addItem = useCallback(async (product, quantity = 1) => {
    if (!isAuthenticated || !user) {
      // Modo local
      setItems(prev => {
        const existing = prev.find(i => i.id === product.id);
        if (existing) {
          return prev.map(i => 
            i.id === product.id 
              ? { ...i, quantity: i.quantity + quantity } 
              : i
          );
        }
        return [...prev, { ...product, quantity }];
      });
      return;
    }

    // Modo persistido
    const { data: existing } = await supabase
      .from('cart_items')
      .select('id, quantity')
      .eq('user_id', user.id)
      .eq('product_id', product.id)
      .maybeSingle();

    if (existing) {
      await supabase
        .from('cart_items')
        .update({ quantity: existing.quantity + quantity })
        .eq('id', existing.id);
    } else {
      await supabase.from('cart_items').insert({
        user_id: user.id,
        product_id: product.id,
        product_name: product.name,
        product_image: product.image,
        product_price: product.price,
        quantity,
      });
    }
    await loadCartItems();
  }, [isAuthenticated, user, loadCartItems]);

  const checkout = useCallback(async (paymentMethod, shippingAddress) => {
    if (!isAuthenticated || !user || items.length === 0) {
      throw new Error('Checkout invalido');
    }

    const total = items.reduce((sum, item) => 
      sum + item.price * item.quantity, 0
    );

    const { data: order } = await supabase
      .from('orders')
      .insert({
        user_id: user.id,
        total,
        status: 'pending',
        payment_method: paymentMethod,
        shipping_address: shippingAddress,
      })
      .select()
      .single();

    await supabase.from('order_items').insert(
      items.map(item => ({
        order_id: order.id,
        product_id: item.id,
        product_name: item.name,
        product_image: item.image,
        product_price: item.price,
        quantity: item.quantity,
      }))
    );

    await supabase
      .from('cart_items')
      .delete()
      .eq('user_id', user.id);

    setItems([]);
    return order;
  }, [isAuthenticated, user, items]);

  return {
    items,
    addItem,
    removeItem,
    updateQuantity,
    clearCart,
    checkout,
    totalItems: items.reduce((sum, i) => sum + i.quantity, 0),
    totalPrice: items.reduce((sum, i) => sum + i.price * i.quantity, 0),
    isLoading,
  };
}
\end{lstlisting}

\subsection{LanguageContext - Internacionalização}

\begin{lstlisting}[language=JavaScript,caption=Sistema de Internacionalização]
// src/context/LanguageContext.tsx
import React, { createContext, useState, useContext, useEffect } from 'react';
import ptBR from '@/locales/pt-br.json';
import en from '@/locales/en.json';

type Language = 'pt-BR' | 'en';
type Translations = typeof ptBR;

interface LanguageContextType {
  language: Language;
  setLanguage: (lang: Language) => void;
  t: (key: string) => string;
}

const translations: Record<Language, Translations> = {
  'pt-BR': ptBR,
  'en': en,
};

const LanguageContext = createContext<LanguageContextType | undefined>(undefined);

export function LanguageProvider({ children }) {
  const [language, setLanguageState] = useState<Language>(() => {
    const saved = localStorage.getItem('language');
    return (saved as Language) || 'pt-BR';
  });

  const setLanguage = (lang: Language) => {
    setLanguageState(lang);
    localStorage.setItem('language', lang);
  };

  const t = (key: string): string => {
    const keys = key.split('.');
    let value: any = translations[language];
    
    for (const k of keys) {
      value = value?.[k];
    }
    
    return value || key;
  };

  return (
    <LanguageContext.Provider value={{ language, setLanguage, t }}>
      {children}
    </LanguageContext.Provider>
  );
}

export const useLanguage = () => {
  const context = useContext(LanguageContext);
  if (!context) throw new Error('useLanguage must be within LanguageProvider');
  return context;
};
\end{lstlisting}

%==============================================================================
\section{Edge Functions para Backend}
%==============================================================================

\subsection{Chatbot com RAG}

\begin{lstlisting}[language=JavaScript,caption=Edge Function do Chatbot]
// supabase/functions/chatbot/index.ts
import { serve } from "https://deno.land/std@0.168.0/http/server.ts";
import { createClient } from "https://esm.sh/@supabase/supabase-js@2";

const corsHeaders = {
  "Access-Control-Allow-Origin": "*",
  "Access-Control-Allow-Headers": "authorization, x-client-info, apikey, content-type",
};

serve(async (req) => {
  if (req.method === "OPTIONS") {
    return new Response("ok", { headers: corsHeaders });
  }

  try {
    const { message, conversationId, sessionId } = await req.json();

    const supabase = createClient(
      Deno.env.get("SUPABASE_URL")!,
      Deno.env.get("SUPABASE_SERVICE_ROLE_KEY")!
    );

    // Buscar entidades similares (RAG)
    const { data: entities } = await supabase
      .from("entity_embeddings")
      .select("entity_name, entity_description, entity_category, entity_price")
      .limit(10);

    // Construir contexto
    const context = entities?.map(e => 
      `- ${e.entity_name}: ${e.entity_description} (${e.entity_category}) - R$ ${e.entity_price}`
    ).join("\n") || "";

    // Chamar IA via Lovable AI Gateway
    const aiResponse = await fetch("https://ai.lovable.dev/v1/chat/completions", {
      method: "POST",
      headers: {
        "Authorization": `Bearer ${Deno.env.get("LOVABLE_API_KEY")}`,
        "Content-Type": "application/json",
      },
      body: JSON.stringify({
        model: "google/gemini-3-flash-preview",
        messages: [
          {
            role: "system",
            content: `Voce e um assistente de [NOME DO NEGOCIO].
            
INFORMACOES DISPONIVEIS:
${context}

DIRETRIZES:
- Responda sempre em portugues
- Seja prestativo e amigavel
- Recomende itens baseado no contexto
- Se nao souber, diga que vai verificar`
          },
          { role: "user", content: message }
        ],
        max_tokens: 1000,
      }),
    });

    const aiData = await aiResponse.json();
    const reply = aiData.choices[0].message.content;

    // Salvar mensagens
    await supabase.from("chat_messages").insert([
      { conversation_id: conversationId, role: "user", content: message },
      { conversation_id: conversationId, role: "assistant", content: reply },
    ]);

    return new Response(
      JSON.stringify({ reply, conversationId }),
      { headers: { ...corsHeaders, "Content-Type": "application/json" } }
    );
  } catch (error) {
    return new Response(
      JSON.stringify({ error: error.message }),
      { status: 500, headers: { ...corsHeaders, "Content-Type": "application/json" } }
    );
  }
});
\end{lstlisting}

%==============================================================================
\section{Componentes de Interface}
%==============================================================================

\subsection{Widget do Chatbot}

\begin{lstlisting}[language=JavaScript,caption=Componente ChatbotWidget]
// src/components/chatbot/ChatbotWidget.tsx
import React, { useState } from 'react';
import { MessageCircle, X, Send } from 'lucide-react';
import { Button } from '@/components/ui/button';
import { supabase } from '@/integrations/supabase/client';

export default function ChatbotWidget() {
  const [isOpen, setIsOpen] = useState(false);
  const [messages, setMessages] = useState([]);
  const [input, setInput] = useState('');
  const [isLoading, setIsLoading] = useState(false);

  const sendMessage = async () => {
    if (!input.trim() || isLoading) return;

    const userMessage = input.trim();
    setInput('');
    setMessages(prev => [...prev, { role: 'user', content: userMessage }]);
    setIsLoading(true);

    try {
      const { data } = await supabase.functions.invoke('chatbot', {
        body: { message: userMessage, sessionId: 'session-id' }
      });

      setMessages(prev => [...prev, { role: 'assistant', content: data.reply }]);
    } catch (error) {
      setMessages(prev => [...prev, { 
        role: 'assistant', 
        content: 'Desculpe, ocorreu um erro.' 
      }]);
    } finally {
      setIsLoading(false);
    }
  };

  return (
    <>
      {/* Botao flutuante */}
      <Button
        onClick={() => setIsOpen(true)}
        className="fixed bottom-6 right-6 rounded-full w-14 h-14 shadow-lg"
      >
        <MessageCircle className="w-6 h-6" />
      </Button>

      {/* Janela do chat */}
      {isOpen && (
        <div className="fixed bottom-24 right-6 w-96 h-[500px] 
                        bg-background border rounded-lg shadow-xl flex flex-col">
          {/* Header */}
          <div className="p-4 border-b flex justify-between items-center">
            <h3 className="font-semibold">Assistente Virtual</h3>
            <Button variant="ghost" size="icon" onClick={() => setIsOpen(false)}>
              <X className="w-4 h-4" />
            </Button>
          </div>

          {/* Mensagens */}
          <div className="flex-1 overflow-y-auto p-4 space-y-4">
            {messages.map((msg, i) => (
              <div key={i} className={`flex ${
                msg.role === 'user' ? 'justify-end' : 'justify-start'
              }`}>
                <div className={`max-w-[80%] p-3 rounded-lg ${
                  msg.role === 'user' 
                    ? 'bg-primary text-primary-foreground' 
                    : 'bg-muted'
                }`}>
                  {msg.content}
                </div>
              </div>
            ))}
          </div>

          {/* Input */}
          <div className="p-4 border-t flex gap-2">
            <input
              value={input}
              onChange={(e) => setInput(e.target.value)}
              onKeyPress={(e) => e.key === 'Enter' && sendMessage()}
              placeholder="Digite sua mensagem..."
              className="flex-1 px-3 py-2 border rounded-lg"
            />
            <Button onClick={sendMessage} disabled={isLoading}>
              <Send className="w-4 h-4" />
            </Button>
          </div>
        </div>
      )}
    </>
  );
}
\end{lstlisting}

%==============================================================================
\section{Estrutura de Arquivos do Projeto}
%==============================================================================

\begin{lstlisting}[caption=Estrutura Completa de Diretórios]
projeto/
├── public/
│   ├── favicon.svg
│   └── robots.txt
├── src/
│   ├── assets/                    # Imagens e recursos
│   ├── components/
│   │   ├── ui/                    # Componentes shadcn/ui
│   │   ├── layout/
│   │   │   ├── Header.tsx
│   │   │   └── Footer.tsx
│   │   ├── home/
│   │   │   ├── HeroSection.tsx
│   │   │   ├── CategorySection.tsx
│   │   │   └── FeaturedProducts.tsx
│   │   ├── product/
│   │   │   └── ProductCard.tsx
│   │   └── chatbot/
│   │       └── ChatbotWidget.tsx
│   ├── context/
│   │   ├── AuthContext.tsx
│   │   ├── CartContext.tsx
│   │   ├── LanguageContext.tsx
│   │   └── ThemeContext.tsx
│   ├── hooks/
│   │   ├── useAuth.ts
│   │   ├── useCartPersistence.ts
│   │   └── useOrders.ts
│   ├── integrations/
│   │   └── supabase/
│   │       ├── client.ts          # Auto-gerado
│   │       └── types.ts           # Auto-gerado
│   ├── locales/
│   │   ├── pt-br.json
│   │   └── en.json
│   ├── pages/
│   │   ├── Index.tsx
│   │   ├── Products.tsx
│   │   ├── ProductDetail.tsx
│   │   ├── Cart.tsx
│   │   ├── Auth.tsx
│   │   ├── OrderHistory.tsx
│   │   ├── Contact.tsx
│   │   ├── Help.tsx
│   │   └── NotFound.tsx
│   ├── types/
│   │   └── product.ts
│   ├── lib/
│   │   └── utils.ts
│   ├── App.tsx
│   ├── main.tsx
│   └── index.css
├── supabase/
│   ├── config.toml
│   ├── functions/
│   │   └── chatbot/
│   │       └── index.ts
│   └── migrations/
│       └── *.sql
├── index.html
├── vite.config.ts
├── tailwind.config.ts
├── tsconfig.json
└── package.json
\end{lstlisting}

%==============================================================================
\section{Configuração do App Principal}
%==============================================================================

\begin{lstlisting}[language=JavaScript,caption=App.tsx com Todos os Providers]
// src/App.tsx
import { Toaster } from "@/components/ui/toaster";
import { TooltipProvider } from "@/components/ui/tooltip";
import { QueryClient, QueryClientProvider } from "@tanstack/react-query";
import { BrowserRouter, Routes, Route } from "react-router-dom";
import { CartProvider } from "@/context/CartContext";
import { AuthProvider } from "@/context/AuthContext";
import { LanguageProvider } from "@/context/LanguageContext";
import { ThemeProvider } from "@/context/ThemeContext";
import ChatbotWidget from "@/components/chatbot/ChatbotWidget";

// Paginas
import Index from "./pages/Index";
import Products from "./pages/Products";
import ProductDetail from "./pages/ProductDetail";
import Cart from "./pages/Cart";
import Auth from "./pages/Auth";
import OrderHistory from "./pages/OrderHistory";
import Contact from "./pages/Contact";
import Help from "./pages/Help";
import NotFound from "./pages/NotFound";

const queryClient = new QueryClient();

const App = () => (
  <QueryClientProvider client={queryClient}>
    <ThemeProvider>
      <LanguageProvider>
        <AuthProvider>
          <CartProvider>
            <TooltipProvider>
              <Toaster />
              <BrowserRouter>
                <Routes>
                  <Route path="/" element={<Index />} />
                  <Route path="/products" element={<Products />} />
                  <Route path="/products/:id" element={<ProductDetail />} />
                  <Route path="/cart" element={<Cart />} />
                  <Route path="/auth" element={<Auth />} />
                  <Route path="/orders" element={<OrderHistory />} />
                  <Route path="/contact" element={<Contact />} />
                  <Route path="/help" element={<Help />} />
                  <Route path="*" element={<NotFound />} />
                </Routes>
                <ChatbotWidget />
              </BrowserRouter>
            </TooltipProvider>
          </CartProvider>
        </AuthProvider>
      </LanguageProvider>
    </ThemeProvider>
  </QueryClientProvider>
);

export default App;
\end{lstlisting}

%==============================================================================
\section{Adaptação para Diferentes Negócios}
%==============================================================================

\subsection{Exemplos de Customização}

\begin{tcolorbox}[colback=yellow!5!white,colframe=orange!75!black,title=Templates por Segmento]

\textbf{E-COMMERCE (Loja Virtual)}
\begin{itemize}[nosep]
    \item Entidades: Produtos, Categorias, Marcas
    \item Páginas extras: Wishlist, Comparador, Rastreamento
    \item Métodos pagamento: PIX, Cartão, Boleto
\end{itemize}

\textbf{RESTAURANTE / DELIVERY}
\begin{itemize}[nosep]
    \item Entidades: Pratos, Categorias, Ingredientes
    \item Páginas extras: Cardápio Digital, Reservas
    \item Integrações: iFood, Rappi, Google Maps
\end{itemize}

\textbf{CLÍNICA / CONSULTÓRIO}
\begin{itemize}[nosep]
    \item Entidades: Serviços, Profissionais, Horários
    \item Páginas extras: Agendamento, Prontuário
    \item Compliance: LGPD, Dados sensíveis
\end{itemize}

\textbf{IMOBILIÁRIA}
\begin{itemize}[nosep]
    \item Entidades: Imóveis, Bairros, Tipos
    \item Páginas extras: Busca avançada, Visita virtual
    \item Integrações: Google Maps, Tour 360
\end{itemize}

\textbf{ESCOLA / CURSOS}
\begin{itemize}[nosep]
    \item Entidades: Cursos, Módulos, Instrutores
    \item Páginas extras: Área do aluno, Certificados
    \item Integrações: Vídeo (Vimeo), Pagamentos recorrentes
\end{itemize}

\end{tcolorbox}

%==============================================================================
\section{Checklist de Implementação}
%==============================================================================

\begin{tcolorbox}[colback=green!5!white,colframe=green!75!black,title=Verificação Final]
\begin{enumerate}[leftmargin=*]
    \item[$\square$] Banco de dados criado com todas as tabelas
    \item[$\square$] RLS habilitado em todas as tabelas
    \item[$\square$] Trigger de criação de perfil funcionando
    \item[$\square$] Autenticação com login e registro
    \item[$\square$] Carrinho sincronizando com banco
    \item[$\square$] Checkout criando pedidos corretamente
    \item[$\square$] Histórico de pedidos exibindo dados
    \item[$\square$] Chatbot respondendo com contexto
    \item[$\square$] Troca de idioma funcionando
    \item[$\square$] Tema claro/escuro alternando
    \item[$\square$] Layout responsivo em mobile
    \item[$\square$] Todas as rotas navegáveis
    \item[$\square$] Tratamento de erros implementado
    \item[$\square$] Loading states visíveis
    \item[$\square$] SEO básico configurado
\end{enumerate}
\end{tcolorbox}

%==============================================================================
\section{Conclusão}
%==============================================================================

Este documento apresenta uma arquitetura completa e replicável para construção de plataformas web modernas. O sistema descrito inclui todas as funcionalidades essenciais para operação de um negócio digital:

\begin{itemize}
    \item \textbf{Segurança}: Autenticação robusta e RLS em banco de dados
    \item \textbf{Escalabilidade}: Arquitetura serverless com Edge Functions
    \item \textbf{Inteligência}: Chatbot com RAG para atendimento automatizado
    \item \textbf{Globalização}: Suporte multilíngue nativo
    \item \textbf{Experiência}: Interface responsiva com temas personalizáveis
\end{itemize}

A combinação de React, TypeScript, Supabase e IA generativa proporciona uma base sólida para qualquer tipo de aplicação web, desde pequenos negócios até grandes corporações.

\end{document}
